% !Mode:: "TeX:UTF-8"
%% This is file `mcmthesis-demo.tex',
%% generated with the docstrip utility.
%%
%% The original source files were:
%%
%% mcmthesis.dtx  (with options: `demo')
%%
%% -----------------------------------
%%
%% This is a generated file.
%%
%% Copyright (C)
%%     2010 -- 2015 by Zhaoli Wang
%%     2014 -- 2015 by Liam Huang
%%
%% This work may be distributed and/or modified under the
%% conditions of the LaTeX Project Public License, either version 1.3
%% of this license or (at your option) any later version.
%% The latest version of this license is in
%%   http://www.latex-project.org/lppl.txt
%% and version 1.3 or later is part of all distributions of LaTeX
%% version 2005/12/01 or later.
%%
%% This work has the LPPL maintenance status `maintained'.
%%
%% The Current Maintainer of this work is Liam Huang.
%%
\documentclass{mcmthesis}
\mcmsetup{tcn = 46364, problem = C,
        sheet = true, titleinsheet = true, keywordsinsheet = false,
        titlepage = true, abstract = true}
%\usepackage{palatino}
\usepackage{times}

\usepackage{lipsum}
\renewcommand{\sfdefault}{ptm}
\title{A Nonlinear Programming Approach for the College Investment Problem}
\author{Zexiang Liu, Mingjian Fu, Yichang Gao}
\date{\today}


%\renewcommand\abstractname{Abstract}


\begin{document}
\begin{abstract}
\lipsum[1-2]

\begin{keywords}
keyword1; keyword2
\end{keywords}
\end{abstract}

\maketitle

%tocloft texdoc tocloft
\tableofcontents

\newpage

\section{Introduction}

\paragraph{} This article aims at proposing one model of the optimal investment strategy for the Goodgrant Foundation. The strategy will explicitly show the donated schools and their corresponding investment amount.This strategy results from the maximum return on the investment(ROI), which is an expectation for the positive effect on the student performance caused by the investment. It is obvious that this parameter is associated with not only the income improvement of the colleges and their graduates benefited from the investment, but also the social welfare. For example, the colleges should ensure the bottom amount for the minority students to promote racial equality. Another example is that the donation for the students from low income families should be keep to certain percentage because the investment can narrow the gap between the rich and poor. However, this kind of value cannot be indicated in any financial increase.
\paragraph{} According to the background, we hope to find the optimal strategy according to the following conditions.
\\ 	1. Maximize the return of the investment.
\\ 	2. Ensure the interests of some certain groups.
\paragraph{} Therefore, our team solves the problem in the following steps.
\\ 	1. Evaluate the effects of the variables in IPEDS data to the return of the investment.
\\  2. Construct the function of the total return for each school of the investment.
\\  3. Find the constraints of the investment amount considering students from certain social groups.
\\  4. Find the optimum solution of the function combining the  constraints.

\begin{itemize}
\item minimizes the discomfort to the hands, or
\item maximizes the outgoing velocity of the ball.
\end{itemize}
We focus exclusively on the second definition.

\begin{itemize}
\item the initial velocity and rotation of the ball,
\item the initial velocity and rotation of the bat,
\item the relative position and orientation of the bat and ball, and
\item the force over time that the hitter hands applies on the handle.
\end{itemize}

\begin{itemize}
\item the angular velocity of the bat,
\item the velocity of the ball, and
\item the position of impact along the bat.
\end{itemize}
\lipsum[4]
\emph{center of percussion} [Brody 1986], \lipsum[5]



%=======
\begin{Theorem} \label{thm:latex}
\LaTeX
\end{Theorem}
\begin{Lemma} \label{thm:tex}
\TeX .
\end{Lemma}
\begin{proof}
The proof of theorem.
\end{proof}


\section{Assumptions}

\paragraph{} 1. The investment only aims at the undergraduates.
\paragraph{} 2. The basic situation of these schools can be almost reflect from the provided data and, generally, the situation will not change.
\paragraph{} 3. The donation is only provided to the colleges whose scorecard data required is complete.


\section{Analysis of the Problem}

%LaTeX插图指南
\begin{figure}[h]
\small
\centering
\includegraphics[width=12cm]{mcmthesis-aaa.eps}
\caption{aa} \label{fig:aa}
\end{figure}

%1,不要用子图,subfig,subfigure。
%2,尽量减少浮动环境,图尽量,缩小图的占位

\eqref{aa}
\begin{equation}
a^2 \label{aa}
\end{equation}

\[
  \begin{pmatrix}{*{20}c}
  {a_{11} } & {a_{12} } & {a_{13} }  \\
  {a_{21} } & {a_{22} } & {a_{23} }  \\
  {a_{31} } & {a_{32} } & {a_{33} }  \\
  \end{pmatrix}
  = \frac{{Opposite}}{{Hypotenuse}}\cos ^{ - 1} \theta \arcsin \theta
\]

\[
  p_{j}=\begin{cases} 0,&\text{if $j$ is odd}\\
  r!\,(-1)^{j/2},&\text{if $j$ is even}
  \end{cases}
\]

\[
  \arcsin \theta  =
  \mathop{{\int\!\!\!\!\!\int\!\!\!\!\!\int}\mkern-31.2mu
  \bigodot}\limits_\varphi
  {\mathop {\lim }\limits_{x \to \infty } \frac{{n!}}{{r!\left( {n - r}
  \right)!}}} \eqno (1)
\]

\section{Linear Model}

\subsection{The Model Results}

\section{Validating the Model}

\section{Conclusions}
\lipsum[6]

\section{A Summary}
\lipsum[6]

\section{Evaluate of the Mode}

\section{Strengths and weaknesses}
\lipsum[12]

\subsection{Strengths}
\begin{itemize}
\item \textbf{Applies widely}\\
This  system can be used for many types of airplanes, and it also
solves the interference during  the procedure of the boarding
airplane,as described above we can get to the  optimization
boarding time.We also know that all the service is automate.
\item \textbf{Improve the quality of the airport service}\\
Balancing the cost of the cost and the benefit, it will bring in
more convenient  for airport and passengers.It also saves many
human resources for the airline. 
\end{itemize}

%(author, 1998)  APA style.

\begin{thebibliography}{99}
\bibitem{1} D.~E. KNUTH   The \TeX{}book  the American
Mathematical Society and Addison-Wesley
Publishing Company , 1984-1986.
\bibitem{2}Lamport, Leslie,  \LaTeX{}: `` A Document Preparation System '',
Addison-Wesley Publishing Company, 1986.
\bibitem{3}\url{http://www.latexstudio.net/}
\bibitem{4}\url{http://www.chinatex.org/}
\end{thebibliography}

%\hspace{2em}
\begin{appendices}

\section{First appendix}

\lipsum[13]

Here are simulation programmes we used in our model as follow.\\

\textbf{\textcolor[rgb]{0.98,0.00,0.00}{Input matlab source:}}
\lstinputlisting[language=Matlab]{./code/mcmthesis-matlab1.m}

\section{Second appendix}

some more text \textcolor[rgb]{0.98,0.00,0.00}{\textbf{Input C++ source:}}
\lstinputlisting[language=C++]{./code/mcmthesis-sudoku.cpp}

\end{appendices}
\end{document}

%%
%% This work consists of these files mcmthesis.dtx,
%%                                   figures/ and
%%                                   code/,
%% and the derived files             mcmthesis.cls,
%%                                   mcmthesis-demo.tex,
%%                                   README,
%%                                   LICENSE,
%%                                   mcmthesis.pdf and
%%                                   mcmthesis-demo.pdf.
%%
%% End of file `mcmthesis-demo.tex'.
